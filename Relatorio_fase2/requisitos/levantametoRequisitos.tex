\chapter{Levantamento e análise de requisitos}

\section{Métodos de levantamento e de análise de requisitos adotados}

Durante a fase de coleta de requisitos do sistema foram usadas diversas técnicas, como a consulta de documentos utilizados (memorandos, emails, queixas de empregados face ao sistema e criticas/avaliação da performance do sistema), para perceber a origem da necessidade de uma base de dados. O dono, futuro administrador da base de dados, listou as funções que acha necessárias e deu o seu parecer acerca do funcionamento normal do ginásio. Foram realizadas entrevistas adicionais com os funcionários do estabelecimento para verificação dos requisitos já identificados, bem como clarificação dos mesmos e possíveis adições de novos requisitos que fossem de encontro às necessidades destes. Por fim, a equipa de levantamento de requisitos procedeu à coleta de dados por observação do normal funcionamento do ginásio através do sistema de vigilância, visto que o facto de um funcionário saber que está a ser observado pode interferir na forma como uma atividade é executada, e como clientes, de modo a ter uma perceção de como a base de dados poderia influenciar a experiência de um cliente.


\section{Requisitos Levantados}

\subsection{Requisitos de Descrição}

Os requisitos de descrição necessários á base de dados recolhidos pela equipa foram :

\begin{enumerate}
    \item  É necessário guardar o nome,sexo,idade, Serviço Prestado, contacto,email e Morada do funcionário. 
    \item Um Serviço é constituído por um nome e pelo respetivo preço.
    \item Temos de guardar todos os dados pessoais do cliente(nome, sexo,idade,contacto,email,morada,Número de Contribuinte) e \textit{IMC}, Peso, Altura, e Limitações Físicas.
    \item Os exercícios são compostos pelo seu tipo e nome associado ao mesmo.
    \item O equipamento tem nome e um propósito. 
    \item A fatura é composta pelo numero de contribuinte empresa, data de emissão, descrição, valor, número contribuinte do cliente, serviço prestado.
    
\item O cliente tem um conjunto de  de exercícios,que formam um plano de exercícios.

\item O plano de exercícios é composto pelo número de repetições e número de series.

\item O Cliente antes de ter acesso aos serviços fornecidos no ginásio tem de se registar nos mesmos,registando a data de subscrição.

\item O ginásio tem de ter a funcionalidade de emitir fatura dos serviços fornecidos ao cliente.

\item Após a subscrição de um ou vários serviços, é submetida uma fatura desses serviços, sendo necessário guardar essas relações para futura consulta.

\item Existem funcionários a fornecer serviços específicos.

\item Deve ser registado o momento em que o funcionário se torna responsável pela prestação do serviço.


\item O cliente tem sempre exercícios, a menos q este se subscreva a um serviço de natação ou a um serviço de nutrição, e estes últimos são sempre diferentes de pessoa para pessoa.

\item Para todo o exercício existe também um equipamento adequado para a sua execução.
\par
\end{enumerate}
\subsection{Requisitos de Exploração}
\label{subsec:requisitos}
Após a verificação das necessidades dos funcionários e do responsável do ginásio para permitir um bom funcionamento deste, chegamos às seguintes conclusões:

\begin{enumerate} 
    \item Terá de ser possível ao responsável do ginásio adicionar um funcionário.
    \item Terá também  de ser possível ao administrador do ginásio alterar a informação relativa a um funcionário;
    \item Para posterior confirmação dos updates ou adições referentes ao funcionário o administrador deverá poder visualizar o conteúdo da tabela dos funcionários.
    \item Tem de ser possível visualizar que serviços são prestados por que funcionários. 
    \item O responsável do ginásio poderá também consultar os serviços fornecidos pelo ginásio.
    \item Outra das funções do administrador do ginásio será adicionar novos serviços que poderão vir a ser disponibilizados aos clientes.
    \item Este terá também a obrigação de marcar como não disponíveis os serviços que entretanto forem descontinuados ou os quais o ginásio já não suporte.
    \item Terá de ser possível ao funcionário registar um novo cliente;
    \item Existe a necessidade de aceder à ficha do cliente pelo número de utilizador;
    \item Também deverá ser possível aceder à ficha do cliente através do seu número de telefone, caso este não saiba o seu idCliente;
    \item Terá de ser possível ao funcionário consultar os exercícios de um cliente.
    \item O funcionário deverá ter permissões para adicionar novos exercícios a um cliente.
    \item O funcionário atualizará mensalmente o IMC do cliente.
    \item Para além disso terá de ser também disponibilizada forma de apagar os exercícios associados a um cliente do seu plano de treinos;
    \item Deverá ser possível a consulta de todas as faturas emitidas pelo ginásio relativas a um cliente.
    \item A consulta de faturas entre  duas datas específicas será também uma funcionalidade presente na base de dados;
    \item Ao funcionário reserva-se o dever de emitir faturas.
    \item Ao funcionário reserva-se também o dever de marcar uma fatura como inválida sempre que esta for mal inserida no sistema.
    \item O responsável reserva-se o dever de ter acesso ao total faturado num determinado período de tempo ou durante todo o tempo de vida do ginásio.
    \item Cabe ao funcionário designar qual o equipamento a ser utilizado num determinado exercício no momento da inserção deste na base de dados.
    \item Quer o administrador quer os rececionistas devem poder ver que serviços foram registados em cada fatura.
    \item Os funcionários devem poder também ver quais as subscrições dos clientes.
    \item No ginásio é requerida a necessidade de poder adicionar novos equipamentos, isto poderá ser feito quer pelos funcionários quer pelo administrador do ginásio.
    \item Um funcionário pode decidir adicionar um novo exercício aos atualmente disponíveis para os clientes.
    \item O administrador será responsável por referenciar que funcionários executam que serviço.
\end{enumerate}


\subsection{Requisitos de Controlo}

\begin{enumerate}
\item Horário de acesso:\par
Tanto o administrador como os funcionários podem aceder à base de dados durante o horário de funcionamento do ginásio, das 8 da manhã até à meia noite.
\item Funcionários e Serviços:\par
Apenas o administrador poderá consultar, adicionar ou remover dados referentes a estas entidades.
\item Cliente:\par
Um funcionário tem total acesso ao cliente podendo adicionar, remover ou modificar os dados de um cliente.
\item Equipamento:\par
Tanto o funcionário como o administrador podem alterar os dados referentes a um equipamento.
\item Exercícios:\par
O funcionário poderá adicionar ou modificar  um exercício.
\item Faturas:\par
Tanto o administrador como o funcionário podem consultar e alterar as faturas. Mas apenas um administrador pode remover uma.

\end{enumerate}

\section{Análise geral dos Requisitos}

Para esta base de dados realizou-se uma combinação das visões do administrador, dos funcionários e dos clientes utilizando a  \textit{Centralized aproach} pois o sistema a modelar não é muito complexo e as vistas dos utilizadores sobrepõem-se em vários aspetos. (Apesar dos clientes não operarem sobre a base de dados, é importante saber as suas necessidades como obter a lista de exercícios correspondentes ao seu plano de treino.)
\newline
\par
Ao realizar uma análise mais profunda dos requisitos podemos inferir que o principal interveniente do sistema é o funcionário visto que este lida com os dados dos clientes, exercícios, faturas e equipamentos do ginásio.
Podemos também verificar que a entidade central é o Cliente pois este realiza exercícios, subscreve um ou mais serviços e paga pelos mesmos.

