\section{Validação do modelo através da normalização}

\subsection{Dependências Funcionais}

\noindent
\\\textbf{Cliente:}
\\ClienteID ®  Nome, Idade, Sexo, Peso, Altura, IMC, Limitações Físicas,  NrContribuinte, idContacto, idMorada
\\NrContribuinte ® ClienteID, Nome, Idade, Sexo, Peso, Altura, IMC, Limitações Físicas, idContacto, idMorada
\\idContacto ® ClienteID, Nome, Idade, Sexo, Peso, Altura, IMC, Limitações Físicas, NrContribuinte, idMorada

\noindent
\\\textbf{Contacto:}
\\idContacto ® Telemóvel 1, Telemóvel 2, Email

\noindent
\\\textbf{Morada:}
\\idMorada ® Rua, Localidade, CodigoPostal

\noindent
\\\textbf{Serviço:}
\\ServiçoID ® Designação, Preço

\noindent
\\\textbf{Fatura:}
\\FaturaID ® Contribuinte Empresa, Data, Desconto, Descrição, Valor, ClienteID, FuncionárioID

\noindent
\\\textbf{Funcionário:}
\\FuncionárioID ® Nome, Cargo, Idade, idContacto, idMorada

\noindent
\\\textbf{ContactoFuncionario:}
\\idContacto ® Telemóvel 1, Telemóvel 2, Email

\noindent
\\\textbf{MoradaFuncionario:}
\\idMorada ® Rua, Localidade, CodigoPostal

\noindent
\\\textbf{Exercício:}
\\ExercícioID ® Designação, Tipo de Exercício

\noindent
\\\textbf{Equipamento:}
\\EquipamentoID ® Descrição, Nome

\noindent
\\\textbf{Subscreve:}
\\ClienteID, ServiçoID ® Data Subscrição

\noindent
\\\textbf{E prestado por:}
\\ServiçoID, FuncionárioID ® Data Início

\noindent
\\\textbf{Plano Exercícios:}
\\ClienteID, ExercicioID ® Numero de Repetições, Numero de Series



\subsection{Primeira Forma Normal (1FN)}

Todas as relações estão na Primeira Forma Normal, pois, para cada relação, a interseção de uma linha e uma coluna contém apenas um valor.

\subsection{Segunda Forma Normal (2FN)}

As relações estão na Segunda Forma Normal, pois cada relação está na 1FN e todos os atributos não chave primária dessa relação dependem da totalidade da chave primária.

\subsection{Terceira Forma Normal (3FN)}

As relações estão na Terceira Forma Normal, pois cada relação está na 1FN e na 2FN e todos os atributos não chave primária dessa relação e dependem diretamente da chave primária, ou seja, não existem dependências transitivas.