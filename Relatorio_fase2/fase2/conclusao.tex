\chapter{Conclusão}
Comecemos por reafirmar que MongoDB ou um sistema de base de dados NoSQL não seria o melhor sistema de base de dados para o nosso projeto, uma vez que está sujeita a muitas operações de escrita e por ser uma base de dados de tamanho reduzido, tornando-se algo redundante a busca por uma eficácia quase insignificante na leitura e consulta dos dados ao custo do desempenho da base de dados em termos de escrita.
\par Olhando para a base de dados resultante da migração para MongoDB como um suporte a base de dados SQL às interrogações do utilizador, podemos usufruir das vantagens das bases de dados orientadas a documentos, consultas simples e mais rápidas, sem abdicarmos do desempenho da base de dados relacional aproveitando o melhor dos dois tipos de base de dados.
\par Com base no conhecimento adquirido ao longo do projeto, chegamos a conclusão que não havia um sistema de base de dados superior a outro nem que viria a substituir, são no entanto duas ferramentas que se podem complementar para ajudar a obter o melhor duma base de dados.
