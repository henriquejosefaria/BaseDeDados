\chapter{Conclusões e Trabalho Futuro}

Neste trabalho construimos uma  base de dados para o ginásio do Sr. Miguel, cumprindo todos os requisitos que o mesmo nos  apresentou. De referir, que este trabalho se tornou mais extenso e trabalhoso do que o expectado, todavia o resultado final compensou a nossa dedicação uma vez que conseguimos alcançar a satisfação do nosso cliente e das metas que nos foram  colocadas. Gostaríamos de realçar ainda que este trabalho contribuiu para que fossemos capaz de desenvolver uma melhor capacidade de trabalho em equipa e de delegação de tarefas.  \par
Durante o desenvolvimento deste projeto, percebemos que o auxilio do motor de base de dados MySQL foi de grande importância visto que nos proporcionou ferramentas fundamentais para facilitar várias das nossas tarefas e garantir toda a consistência dos nossos dados.

\par O nosso maior desafio, mas também uma conquista foi conseguir chegar a um consenso no nosso modelo conceptual. Numa primeira etapa do desenvolvimento do projeto, o modelo parecia-nos correto, contudo tal não se verificou o que nos levou a investir uma grande quantidade de tempo a remodelar o mesmo para que tudo estivesse de acordo com o que o nosso cliente desejava e de acordo com as regras de normalização e das etapas da modelação lógica.\par
Para além disso, é de salientar que todo o processo de modelação, desde modelo conceptual até ao físico, é imprescindível na criação de uma base de dados para promover a organização e consistência de todo o processo. De facto, este método contribuiu imenso para o nosso sucesso.
\par Numa perspetiva futura, temos como objetivo implementar uma entidade Fornecedor que estaria encarregue da distribuição dos equipamentos e no caso de um equipamento avariar seria possível chegar ao fornecedor do mesmo e comunicar a necessidade da reparação do equipamento. Além disso queremos expandir a base de dados para mais ginásios, ou seja, criar uma entidade ginásio, permitindo nos comercializar facilmente a base de dados.
