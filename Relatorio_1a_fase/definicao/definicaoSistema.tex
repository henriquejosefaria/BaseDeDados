\chapter{Definição do Sistema}
\section{Contexto da Aplicação do Sistema}

O Miguel gosta muito de fazer exercício. Ultrajado com o preço dos ginásios na sua área, ele decide investir no seu próprio equipamento e começar um pequeno ginásio pessoal. Sendo um estudante de Engenharia Informática extremamente popular entre o público feminino, começou a receber diversos pedidos das suas amigas para também poderem treinar no seu ginásio. Sabendo disto, alguns outros estudantes demonstraram interesse em treinar no ginásio do Miguel. 

\section{ Fundamentação da Implementação da Base de Dados}
 Com o aumento de clientes, o ginásio teve a necessidade de expandir as suas instalações, comprar novo equipamento e também recrutar novos funcionários, para conseguir satisfazer a necessidade dos seus clientes.
 Como tal, Miguel concluiu que era conveniente armazenar a informação detalhada dos seus funcionários e equipamentos.
 \par Com a intenção de criar um ligação mais próxima com os clientes, o Miguel intendeu que se devia guardar também a informação do cliente. Uma ação como mandar um postal pelo Natal, ou pelo aniversário e mandar mensagens para incentivar o treino era impossível sem guardar os dados dos clientes.
\section{Análise de Viabilidade do Processo}
Trata-se de um modelo genérico de um sistema de base de dados, desenvolve-se em torno de um fornecedor(O Miguel), serviço(todas as atividades possíveis desenvolver no ginásio, desde personal training a planos de nutrição) e cliente, ou seja, o processo de desenvolvimento desta base de dados é bastante simples e como tal parece-nos um sistema de base de dados fácil de implementar.
