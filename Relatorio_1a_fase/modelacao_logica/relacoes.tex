\section{Derivar relações para o modelo de dados lógicos}

Começamos por derivar relações a partir do modelo concetual seguindo as regras adequadas, de modo a representar no modelo lógico as entidades, relacionamentos e atributos identificados previamente. Tal gerou as seguintes relações. No entanto, para as relações Cliente e Funcionário, decidimos separar os atributos compostos \emph{Contacto} e \emph{Morada} nas suas próprias relações, em vez de colocar os respetivos atributos atómicos nas relações mencionadas, de forma a facilitar a compreensão.


\noindent
\\\textbf{Cliente} (ClienteID,  Nome, Idade, Sexo, Peso, Altura, IMC, Limitações Físicas,  NrContribuinte, idContacto, idMorada)
\\\textbf{Primary Key} ClienteID
\\\textbf{Foreign Key} idContacto \textbf{references} Contacto(idContacto)
\\\textbf{Foreign Key} idMorada \textbf{references} Morada(idMorada)

\noindent
\\\textbf{Contacto} (idContacto, Telemóvel 1, Telemóvel 2, Email)
\\\textbf{Primary Key} idContacto

\noindent
\\\textbf{Morada} (idMorada, Rua, Localidade, CodigoPostal)
\\\textbf{Primary Key} idMorada

\noindent
\\\textbf{Serviço} (ServiçoID, Designação, Preço)
\\\textbf{Primary Key} ServiçoID

\noindent
\\\textbf{Fatura} (FaturaID, Contribuinte Empresa, Data, Desconto, Descrição, Valor, ClienteID, FuncionárioID)
\\\textbf{Primary Key} FaturaID
\\\textbf{Foreign Key} ClienteID \textbf{references} Cliente(ClienteID)
\\\textbf{Foreign Key} FuncionárioID \textbf{references} Funcionário(FuncionárioID)

\noindent
\\\textbf{Funcionário} (FuncionárioID, Nome, Cargo, Idade, idContacto, idMorada)
\\\textbf{Primary Key} FuncionárioID
\\\textbf{Foreign Key} idContacto \textbf{references} ContactoFuncionario(idContacto)
\\\textbf{Foreign Key} idMorada \textbf{references} MoradaFuncionario(idMorada)

\noindent
\\\textbf{ContactoFuncionario} (idContacto, Telemóvel 1, Telemóvel 2, Email)
\\\textbf{Primary Key} idContacto

\noindent
\\\textbf{MoradaFuncionario} (idMorada, Rua, Localidade, CodigoPostal)
\\\textbf{Primary Key} idMorada

\noindent
\\\textbf{Exercício} (ExercícioID, Designação, Tipo de Exercício)
\\\textbf{Primary Key} ExercícioID

\noindent
\\\textbf{Equipamento} (EquipamentoID, Descrição, Nome)
\\\textbf{Primary Key} EquipamentoID

% *:* relationship
\noindent
\\\textbf{Constitui} (ServiçoID, FaturaID)
\\\textbf{Primary Key} ServiçoID, FaturaID
\\\textbf{Foreign Key} ServiçoID \textbf{references} Serviço(ServiçoID)
\\\textbf{Foreign Key} FaturaID \textbf{references} Fatura(FaturaID)

% *:* relationship
\noindent
\\\textbf{Subscreve} (ClienteID, ServiçoID, Data Subscrição)
\\\textbf{Primary Key} ClienteID, ServiçoID
\\\textbf{Foreign Key} ClienteID \textbf{references} Cliente(ClienteID)
\\\textbf{Foreign Key} ServiçoID \textbf{references} Serviço(ServiçoID) 

% *:* relationship
\noindent
\\\textbf{E prestado por} (ServiçoID, FuncionárioID, Data Início)
\\\textbf{Primary Key} ServiçoID, FuncionárioID
\\\textbf{Foreign Key} ServiçoID \textbf{references} Serviço(ServiçoID)
\\\textbf{Foreign Key} FuncionárioID \textbf{references} Funcionário(FuncionárioID)

% *:* relationship
\noindent
\\\textbf{Plano Exercícios} (ClienteID, ExercicioID, Numero de Repetições, Numero de Series)
\\\textbf{Primary Key} ClienteID, ExercicioID
\\\textbf{Foreign Key} ClienteID \textbf{references} Cliente(ClienteID)
\\\textbf{Foreign Key} ExercicioID \textbf{references} Exercício(ExercicioID)

% *:* relationship
\noindent
\\\textbf{É utilizado para} (EquipamentoID, ExercícioID)
\\\textbf{Primary Key} EquipamentoID, ExercícioID
\\\textbf{Foreign Key} EquipamentoID \textbf{references} Equipamento(EquipamentoID)
\\\textbf{Foreign Key} ExercícioID \textbf{references} Exercício(ExercícioID)

